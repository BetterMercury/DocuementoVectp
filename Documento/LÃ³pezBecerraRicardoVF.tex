\documentclass[12pt]{article}
\usepackage[spanish]{babel}
\usepackage{graphicx}
\usepackage{amsmath}
\usepackage{amssymb}
\usepackage{cancel}
\usepackage{hyperref}
\usepackage{xcolor}
\usepackage{esint}
\usepackage{pgfplots}



\graphicspath{ {Imagenes/} }

\pagestyle{empty}
\topmargin = 0pt
\headheight= 0pt
\headsep=0pt
\oddsidemargin= 25pt
\textwidth=420pt
\setlength{\parindent}{0pt}

\DeclareMathOperator{\arcsec}{arcsec}
\DeclareMathOperator{\arccot}{arccot}
\DeclareMathOperator{\arccsc}{arccsc}
\DeclareMathOperator{\di}{d\!}

\date{5 de febrero de 2021}
\title{Serie 4: Cálculo Vectorial}
\author{Ricardo López \and Dante Argüello \and Eugenio \and Ricardo Ruelas}

\begin{document}
	
\maketitle
	
\newcommand{\Int}{\int\limits}

\newcommand*\eval[3]{\left.#1\right\rvert_{#2}^{#3}}

\noindent \textbf{Ejercicio 1:} Calcular $\int_0^1 \int_0^2 xy \,dy\,dx$ .

\vspace{5mm}

\noindent \textbf{Solución.}
 
\vspace{5mm}

\begin{align*}
\int_0^1 \int_0^2 xy \,dy\,dx = \int_0^1 \eval{\frac{xy^2}{2}}{2}{0} \di x 
= \int_0^1 2x \di x 
= \color {red} 1
\end{align*}



\noindent \textbf{Ejercicio 6:} Por medio de la integral doble, calcular el área de la región localizada entre las curvas de ecuación $2y^2=x-2$, $x^2-4y^2=4$, $x=4$.

\vspace{5mm}

\noindent \textbf{Solución.}

\vspace{5mm}

\noindent Despejamos las ecuaciones de las curvas:

\begin{equation}\tag{1}
	2y^2 = x-2 \implies y^2 = \frac{x}{2}-1 \implies y = \sqrt{\frac{x}{2}-1}
\end{equation}

\begin{equation}\tag{2}
	x^2-4y^2=4 \implies 4y^2=x^2-4 \implies y^2=\frac{x^2}{4}-1 \implies \boldsymbol{y=\sqrt{\frac{x^2}{4}-1}} 
\end{equation}

\noindent Obtenemos el valor de $x$ igualando ambas ecuaciones y resolviendo:

\begin{equation}\tag{3}
	\sqrt{\frac{x}{2}-1}=\sqrt{\frac{x^2}{4}-1} \implies \frac{x}{2}-\cancel{1}=\frac{x^2}{4}-\cancel{1} \implies 2x = x^2 \therefore \boldsymbol{x=2}
\end{equation}

\noindent Calculamos la doble integral para obtener el area:

\begin{equation}\tag{4}
	\Int_2^4 \Int_{\sqrt{\frac{x}{2}-1}}^{\sqrt{\frac{x^2}{4}-1}} \, \mathrm{d}y \,\mathrm{d}x \implies \Int_2^4 \left(\sqrt{\frac{x}{2}-1} - \sqrt{\frac{x^2}{4}-1}\right)\, \mathrm{d}x
\end{equation}

\noindent Aplicamos linealidad

\begin{equation}\label{eqn:5}\tag{5}
	\frac{1}{2} \Int_2^4 \sqrt{x^2-4}\, \mathrm{d}x - \frac{1}{\sqrt{2}} \Int_2^4 \sqrt{x-2}\, \mathrm{d}x
\end{equation}

\noindent Resolvemos las integrales por separado:

\begin{equation}\label{eqn:6}\tag{6}
	\frac{1}{2} \Int_2^4 \sqrt{x^2-4}\, \mathrm{d}x
\end{equation}

\begin{equation}\label{eqn:7}\tag{7}
	\frac{1}{\sqrt{2}} \Int_2^4 \sqrt{x-2}\, \mathrm{d}x
\end{equation}

\noindent Realizando sustitución trigonométrica:

\begin{align*}
	x = 2 \sec(u) \to u = \arcsec(\frac{x}{2})\, , \ dx=2\sec(u)\tan(u) \mathrm{d}u
\end{align*}

\noindent Sustituimos en \eqref{eqn:6}:

\begin{align*}
	\frac{1}{2} \int 2\sec(u)\sqrt{4\sec^2(u)-4}\tan(u) \mathrm{d}u
\end{align*}

\noindent Simplificamos usando $4\sec^2(u)-4 = 4\tan^2(u)$:

\begin{align*}
	\frac{1}{2}\, 4\int \sec(u)\tan^2(u)\, \mathrm{d}u \implies 2\int \sec(u)\tan^2(u)\, \mathrm{d}u
\end{align*}

\noindent Reescribimos usando identidades trigonométricas: $\tan^2(u)=\sec^2(u)-1$

\begin{align*}
	= 2\int \sec(u)(\sec^2(u)-1)\, \mathrm{d}u \implies = 2\int (\sec^3(u)-\sec(u))\, \mathrm{d}u
\end{align*}

\begin{align*}
	= 2\left(\int \sec^3(u)\, \mathrm{d}u - \int \sec(u)\, \mathrm{d}u\right)
\end{align*}

\noindent Aplicamos la formula de reducción, con $n=3$:

\begin{center}
	$\int \sec^n(u)\, \mathrm{d}u = \frac{n-2}{n-1} \int \sec^{n-2}(u)\, \mathrm{d}u + \frac{\sec^{n-2}(u)\tan(u)}{n-1}$
\end{center}

\begin{align*}
	= 2\left(\frac{\sec(u)\tan(u)}{2} + \frac{1}{2} \int \sec(u)\, \mathrm{d}u - \int \sec(u)\, \mathrm{d}u\right)
\end{align*}

\begin{align*}
	= 2\left(\frac{\sec(u)\tan(u)}{2} - \frac{1}{2}\int \sec(u)\, \mathrm{d}u\right) \implies 2\left(\frac{\sec(u)\tan(u)}{2} - \frac{1}{2}\ln(\tan(u)+\sec(u))\right)
\end{align*}

\begin{align*}
	= \sec(u)\tan(u) - \ln(\tan(u)+\sec(u))
\end{align*}

\noindent Deshacemos la sustitución $u = \arcsec(\frac{x}{2})$, usando:

\begin{align*}
	\tan\left(\arcsec\left(\frac{x}{2}\right)\right)=\sqrt{\frac{x^2}{4}-1}\quad \mathrm{y} \
	\sec\left(\arcsec\left(\frac{x}{2}\right)\right)=\frac{x}{2}
\end{align*}

\begin{align*}
	\left.\frac{x}{2}\sqrt{\frac{x^2}{4}-1}-\ln\left(\sqrt{\frac{x^2}{4}-1}+\frac{x}{2}\right)\right|_2^4
\end{align*}

\begin{align*}
	=\left(\frac{4}{2}\sqrt{\frac{4^2}{4}-1} - \ln(\sqrt{\frac{4^2}{4}-1}+\frac{4}{2})\right) - \left(\frac{2}{2}\cancelto{0}{\sqrt{\frac{2^2}{4}-1}} - \ln(\cancelto{0}{\sqrt{\frac{2^2}{4}-1}}+\cancelto{1}{\frac{2}{2}})\right)
\end{align*}

\begin{align*}
	=\left(2\sqrt{3} - \ln(\sqrt{3}+2)\right) - \cancelto{0}{\ln(1)}
\end{align*}

\noindent Por lo tanto tenemos que el resultado de la ecuación \eqref{eqn:6} es:

\begin{align*}
	=2\sqrt{3} - \ln(\sqrt{3}+2)
\end{align*}

\noindent Ahora pasamos a resolver resolvemos la ecuación \eqref{eqn:7}:

\begin{align*}
	\frac{1}{\sqrt{2}} \Int_2^4 \sqrt{x-2}\, \mathrm{d}x
\end{align*}

\noindent Utilizamos la sustitución $u=x-2 \to \mathrm{d}u = \mathrm{d}x$:

\begin{align*}
	\int \sqrt{u}\, \mathrm{d}u \implies \int u^{\frac{1}{2}}\, \mathrm{d}u \implies \frac{2}{3}u^{\frac{3}{2}} + C
\end{align*}

\noindent Deshacemos la sustitución previa:

\begin{align*}
	\frac{1}{\sqrt{2}}\left[\left.\frac{2}{3}(x-2)^{\frac{3}{2}}\right|_2^4\right] \implies \frac{2}{3\sqrt{2}} \left[(4-2)^{\frac{3}{2}} - \cancelto{0}{(2-2)^{\frac{3}{2}}}\right] \implies \frac{2}{3\sqrt{2}} \left[(2)^{\frac{3}{2}}\right] = \frac{2\sqrt{8}}{3\sqrt{2}} = \frac{4\cancel{\sqrt{2}}}{3\cancel{\sqrt{2}}} = \pmb{\frac{4}{3}}
\end{align*}

\noindent Resolvemos el problema sustituyendo los valores obtenidos de \eqref{eqn:6} y \eqref{eqn:7} en la ecuación \eqref{eqn:5}:

\begin{align}
	\frac{1}{2} \Int_2^4 \sqrt{x^2-4}\, \mathrm{d}x - \frac{1}{\sqrt{2}} \Int_2^4 \sqrt{x-2}\, \mathrm{d}x = 2\sqrt{3} - \ln(\sqrt{3}+2) - \frac{4}{3}
\end{align}

\begin{align*}
	 \pmb{\color{red} \therefore\: A = 2\sqrt{3} - \ln(\sqrt{3}+2) - \frac{4}{3}\quad u^2}
\end{align*}

\noindent \textbf{Ejercicio 8:} Por medio de la integral doble, calcular el área
de la región localizada entre las curvas de ecuación $x^2 -14x -5y +59 = 0$ , 
$x^2-14x+5y-11=0$.

\vspace{5mm}

\noindent \textbf{Solución.}
 
\vspace{5mm}

Empezamos obteniendo los puntos de intersección de las cruvas

\begin{align*}
	x^2 -14x -5y +59 = x^2-14x+5y-11 \implies y = 7
\end{align*}

\begin{align*}
	&x^2 -14x -5y +59 = 0 \\
	&x^2 -14x -5(7) +59 = 0 \\
	&x = \frac{14 \pm \sqrt{196^2-96 }}{2} \\
	&x_1 = 2 \\
	&x_2 = 12 \\
\end{align*}

Despejando y de ambas ecuaciones

\begin{align*}
	c_1 &= \frac{(x-y)^2}{5} + 2 \\
	c_2 &= -\frac{(x-7)^2}{5}+12
\end{align*}

Para calcular el área se resuelve la integral 

\begin{align*}
	A = \int_{x_1}^{x_2} \int_{c_1}^{c_2} \di A = \int_{2}^{12} \int_{\frac{(x-y)^2}{5} + 2}^{-\frac{(x-7)^2}{5}+12} \di y \di x
\end{align*}

Se eligieron estos límites de integración porque en este caso se debe integrar desde el menor
valor de x al mayor y desde la curva que limita al área por debajo hasta la que la limita
por arriba
\begin{align*}
A &= \int_{2}^{12} \int_{\frac{(x-y)^2}{5} + 2}^{-\frac{(x-7)^2}{5}+12} \di y \di x 
= \int_2^12 \eval{x}{\frac{(x-y)^2}{5} + 2}{-\frac{(x-7)^2}{5}+12} \di x \\
&= \int_2^12 -\frac{2}{5}(x-7)^2+10 \di x = \eval{-\frac{2}{15}(x-7)^3+10x}{2}{12}
=\color{red}\frac{200}{3} u^3 
\end{align*}

\noindent \textbf{Ejercicio 9:} Calcular $\iint\limits_R \frac{x+y}{1+x-y}\, \mathrm{d}A$, donde $R$ es la región limitada por las gráficas de $y=x$, $y=-x$, $y=x-4$ y $y=-x+4$.

\vspace{3mm}

\noindent Sugerencia: Haga la transformación $\begin{cases} u = x+y\\v=x-y\end{cases}$

\noindent \textbf{Solución}.

\vspace{3mm}

\noindent Considerando la transformación sugerida, obtenemos los factores de escala tales que: $\mathrm{d}A = h_u\,h_v\,\mathrm{d}u\,\mathrm{d}v$

\begin{align*}
	\begin{cases}
		u = x+y\\
		v=x-y
	\end{cases} \implies u+v=x+x+\cancel{y}-\cancel{y} \implies 2x = u+v \implies x = \frac{u+v}{2}
\end{align*}

\begin{align*}
	\begin{cases}
		u = x+y\\
		-v = -x+y
	\end{cases} \implies u-v=\cancel{x}-\cancel{x}+y-y \implies 2y = u-v \implies y = \frac{u-v}{2}
\end{align*}

\begin{equation}\label{eqn:9-1}\tag{1}
	\overline{R}(u,v)=\left(\frac{u+v}{2}\right)\hat{i} + (\frac{u-v}{2})\hat{j}
\end{equation}

\noindent Sabiendo que $h_u = |\frac{\partial \overline{R}}{\partial u}|$ y $h_v = |\frac{\partial \overline{R}}{\partial v}|$, entonces:

\begin{align*}
	\frac{\partial \overline{R}}{\partial u} = \frac{1}{2}\hat{i} + \frac{1}{2}\hat{j} \qquad \qquad
	\frac{\partial \overline{R}}{\partial v} = \frac{1}{2}\hat{i} - \frac{1}{2}\hat{j}
\end{align*}

\begin{align*}
	\left|\frac{\partial \overline{R}}{\partial u}\right| = \left|\frac{\partial \overline{R}}{\partial v}\right| = \sqrt{\frac{1}{2}}
\end{align*}

\begin{align*}
	\mathrm{d}A = \sqrt{\frac{1}{2}}\,\sqrt{\frac{1}{2}}\, \mathrm{d}u\,\mathrm{d}v\, \qquad \therefore \qquad \mathrm{d}A = \frac{1}{2}\,\mathrm{d}u\,\mathrm{d}v = \frac{1}{2}\,\mathrm{d}v\,\mathrm{d}u
\end{align*}

\noindent Calculamos los límites de integración a partir de las curvas dadas y la transformación sugerida:

\begin{align*}
	y = x  \implies \cancelto{{\color{blue}u}}{x-y} = 0 \implies {\color{blue}u} = 0 \\
	y = -x \implies \cancelto{{\color{red}v}}{x+y} = 0 \implies {\color{red}v} = 0 \\
	y = x - 4 \implies \cancelto{{\color{blue}u}}{x-y} = 4 \implies {\color{blue}u} = 4 \\
	y = -x + 4 \implies \cancelto{{\color{red}v}}{x+y} = 4 \implies {\color{red}v} = 4 \\
\end{align*}

\noindent Realizamos la integral doble, sustituyendo la transformación donde es posible:

\begin{equation*}
	 \iint\limits_R \frac{x+y}{1+x-y}\, \mathrm{d}A = \Int_0^4\Int_0^4 \frac{u}{1+v}\, \frac{1}{2}\,\mathrm{d}v\,\mathrm{d}u = \frac{1}{2}\,\Int_0^4\Int_0^4 \frac{u}{1+v}\, \mathrm{d}v\,\mathrm{d}u 
\end{equation*}

\noindent Es importante notar que al tener límites de integración iguales en la doble integral, se pueden intercambiar los diferenciales sin mayor complicación. Resolvemos:

\begin{equation}\label{eqn:9-2}\tag{2}
	\frac{1}{2}\,\Int_0^4\Int_0^4 \frac{u}{1+v}\, \mathrm{d}v\,\mathrm{d}u
\end{equation}

\begin{align*}
	= \frac{1}{2}\,\Int_0^4 u \Int_0^4 \frac{\mathrm{d}v}{1+v}\, \mathrm{d}u
	= \frac{1}{2}\,\Int_0^4 u \left.\ln(1+v)\right|_0^4\, \mathrm{d}u = \frac{1}{2}\,\Int_0^4 u( \ln(5)-\cancelto{0}{\ln(1)})\, \mathrm{d}u \\
	= \frac{1}{2}\,\Int_0^4 u \ln(5)\, \mathrm{d}u = \frac{\ln(5)}{2}\,\Int_0^4 u\, \mathrm{d}u = \frac{\ln(5)}{2}\, \left.\frac{u^2}{2}\right|_0^4 = \frac{\ln(5)}{\cancel{4}}\left(4\cancel{^2}-0^2\right)
\end{align*}

\begin{align*}
	\pmb{\color{red} \therefore \iint\limits_R \frac{x+y}{1+x-y}\, \mathrm{d}A\, = \,4\ln(5)}
\end{align*}

\noindent \textbf{Ejercicio 10:} Por medio de la integral doble, calcular el área de la 
región del primer cuadrante, limitada por las curvas $xy = 1$, $xy = 4$, $y = 2x$, $x = 2y$ .

\vspace{5mm}

\noindent \textbf{Solución.}

\vspace{3mm}
Primero se reescriben las ecuaciones de las curvas

\begin{align*}
	y = \frac{1}{x} && y = \frac{4}{x} && y = 2x && y = \frac{x}{2}
\end{align*}

Luego, como apoyo, se grafican las curvas para reconocer los límites de integración
\begin{figure}
\centering
\begin{tikzpicture}
	\begin{axis}[
		axis lines = left,
		xlabel = $x$,
		ylabel = {$y$},
		xmin=0, xmax=4,
		ymin=0, ymax=4,
	]
	
	\addplot [
		domain=-0:4, 
		samples=100, 
	]
	{2*x};

	\addplot [
		domain=0:4, 
		samples=100, 
		]
		{x/2};

	\addplot [
		domain=0:4, 
		samples=100, 
		]
		{4/x};
	
	\addplot [
		domain=0:4, 
		samples=100, 
		]
		{1/x};
	\end{axis}
	\end{tikzpicture}
	\caption{Gráfica de las curvas.}
	\label{ej10}
\end{figure}

De la figura \ref{ej10} se deduce que se necesitan encontrar cuatro puntos de intersección,
por lo que se optienen el valor de x los siguentes sistemas de ecuaciones:
\begin{align*}
	&\begin{cases}
		y &= 2x \\
		y &= \frac{1}{x}
	\end{cases}	
	\implies x = \frac{\sqrt{2}}{2}	\\
	&\begin{cases}
		y &= 2x \\
		y &= \frac{4}{x}
	\end{cases}	
	\implies x = \sqrt{2} \\
	&\begin{cases}
		y &= \frac{x}{2} \\
		y &= \frac{1}{x}
	\end{cases}	
	\implies x = \sqrt{2} \\
	&\begin{cases}
		y &= \frac{x}{2} \\
		y &= \frac{4}{x}
	\end{cases}	
	\implies x = 2\sqrt{2} \\
\end{align*}

Con los puntos obtenidos, se llega a que el área debe ser

\begin{align*}
	A &= \int_{\frac{\sqrt{2}}{2}}^{\sqrt{2}}\int_{\frac{1}{x}}^{2x} \di y \di x
	+ \int_{\sqrt{2}}^{2\sqrt{2}}\int_{\frac{x}{2}}^{\frac{4}{x}} \di y \di x \\
	&=\left(2-\frac{1}{2}-\ln{\sqrt{2}}+ \ln{\frac{\sqrt{2}}{2}}\right) 
	+\left(4\ln{2\sqrt{2}}-4\ln{\sqrt{2}}-\frac{3}{2}\right) \\
	&=\color{red}\ln(8) u^2
\end{align*}

\noindent \textbf{Ejercicio 12} Determinar el volumen de la región limitada por
 las superficies: $az = y^2$, $x^2+y^2 = r2$, $z=0$ donde a y r son constantes.

\vspace{5mm}

\noindent \textbf{Solución.}

\vspace{3mm}

Como una de las curvas es una circunferencia en el plano xy y se quiere calcular un
volumen en que está involucrada, se utilizan coordenadas cilindricas para faciliar
los cálculos, por lo que 

\begin{align*}
	V = \int\int_{R} \frac{y^2}{a} \di A
\end{align*}
donde
\begin{align*}
	\di A = h_{r}h_{\theta} \di r \di \theta && h_{r}=1 && h_{\theta} = r
\end{align*} 
Como se debe considerar el volumen del cilindro completo se debe integrar desde
$0$ hasta $r$ con respecto a $r$ y desde $0$ hasta $2\pi$ con 
respecto a $\theta$, por lo que el volumen se define
como
\begin{align*}
	V = \int_{0}^{2\pi}\int_{0}^{r} \frac{r^2\sen^2\theta}{a}r \di r \di \theta
	= \int_0^{2\pi} \frac{r^4\sen^2\theta}{4a} \di \theta =\color{red} \frac{r^4\pi}{4a} u^3	
\end{align*} 




\noindent \textbf{Ejercicio 13:} Calcular el volumen de la región que es limitada por las superficies $S_1$ y $S_2$ representadas por: $S_1:x^2+z^2=4-y$, $S_2:y+5=0$.

\vspace{5mm}

\noindent \textbf{Solución.}

\vspace{3mm}

\noindent Para calcular el volumen de un sólido dada una región, empleamos una integral triple de volumen en $xyz$:

\begin{equation}\label{eqn:13-1}\tag{1}
	V = \iiint_V dV \implies V = \int_{a}^{b}\int_{g_1(x)}^{g_2(x)}\int_{h_1(x,y)}^{h_2(x,y)}\, \mathrm{d}z\,\mathrm{d}y\,\mathrm{d}x
\end{equation}

\noindent Al aplicar la primera integral, obtenemos justamente la integral doble que se utiliza para calcular \textbf{el volumen entre dos superficies}:

\begin{equation}\label{eqn:13-2}\tag{2}
	V = \int_{a}^{b}\int_{g_1(x)}^{g_2(x)}\, h_2(x,y)-h_1(x,y)\, \mathrm{d}A
\end{equation}

\noindent Notamos que tenemos es más sencillo trabajar con una función $h(x,z)$ debido a que la superficie $S_2$ ya nos da uno de los límites de integración en $y$, así podemos despejar $y$ de la primera superficie para obtener el segundo límite:

\begin{align*}
	S_1:x^2+z^2=4-y \implies x^2+z^2+y=4 \implies \boldsymbol{y=4-x^2-z^2}
\end{align*}

\noindent De esta forma podemos definir lo siguiente:

\begin{equation*}
	h_1(x,z) = -5 \qquad
	h_2(x,z) = 4-x^2-z^2
\end{equation*}

\begin{align*}
	\implies h_2(x,z) - h_1(x,z) = 4-x^2-z^2 - (-5) = 4-x^2-z^2 + 5\\
	\therefore \quad \boldsymbol{h_2(x,z) - h_1(x,z) = 9-x^2-z^2}
\end{align*}

\noindent Sustituimos la función obtenida en la ecuación \eqref{eqn:13-2}:

\begin{align*}
	V = \int_{a}^{b}\int_{g_1(x)}^{g_2(x)}\, 9-x^2-z^2\, \mathrm{d}A
\end{align*}

\noindent A partir de la ecuación anterior, podemos darnos cuenta de que nos conviene realizar una transformación a coordenadas polares, tal que $x^2+z^2 = r^2$ y $\mathrm{d}A = r\, \mathrm{d}r\,\mathrm{d}\theta
$. Además, notamos que se tiene una circunferencia de radio 3, por lo que $a=0,\, b=2\pi,\, g_1(x)=0,\, g_2(x)=3$. Sustituimos en la integral y nos queda:

\begin{align*}
	V = \int_{0}^{2\pi}\int_{0}^{3}\, (9-\underbrace{(x^2+z^2)}_{\text{$r^2$}})\, r\, \mathrm{d}r\,\mathrm{d}\theta = \int_{0}^{2\pi}\int_{0}^{3}\, (9-r^2)\, r\, \mathrm{d}r\,\mathrm{d}\theta = \int_{0}^{2\pi}\int_{0}^{3}\, (9r-r^3)\, \mathrm{d}r\,\mathrm{d}\theta
\end{align*}

\begin{align*}
	= \int_{0}^{2\pi}\, \left.\left(\frac{9r^2}{2}-\frac{r^4}{4}\right)\right|_0^3\ \mathrm{d}\theta = \int_{0}^{2\pi}\, \left(\frac{81}{2}-\frac{81}{4}\right)\,\mathrm{d}\theta = \frac{81}{4}\, \int_{0}^{2\pi}\, \mathrm{d}\theta = \frac{81}{4}\,(2\pi)
\end{align*}

\begin{align*}
	\pmb{\color{red} \therefore\: V = \frac{81}{2}\pi\: u^3}
\end{align*}

\noindent \textbf{Ejercicio 16:} Utilizar la integración doble para calcular el área de la región interior a lac urva cuya ecuación polar es $r=6\cos\theta$.

\vspace{5mm}

\noindent \textbf{Solución.}

\vspace{3mm}

\noindent Notamos que la ecuación polar dada corresponde a una circunferencia con centro en el punto $(3, 0)$ y radio $r=3$. Se sabe que el cálculo de un area por medio de una integral doble esta dada por:

\begin{equation*}
	\iint_R\, \mathrm{d}A
\end{equation*}

\noindent Tomando en cuenta que en coordenadas polares $\mathrm{d}A = r\,\mathrm{d}r\,\mathrm{d}\theta$ y que los límites de integración vienen dados por $r_0 = 0$, $r = 3$, $\theta_0 = 0$ y $\theta = 2\pi$, sustituimos y resolvemos la integral doble:

\begin{align*}
	\Int_0^{2\pi} \Int_0^3\, r\,\mathrm{d}r\,\mathrm{d}\theta = \Int_0^{2\pi} \left.\left(\frac{r^2}{2}\right)\right|_0^3\mathrm{d}\theta = \Int_0^{2\pi} \left(\frac{3^2}{2}\right)\,\mathrm{d}\theta = \frac{9}{2}\Int_0^{2\pi} \,\mathrm{d}\theta = \frac{9}{2}\left(2\pi - 0\right) = \boldsymbol{9\pi}
\end{align*}

\begin{align*}
	\pmb{\color{red} \therefore\: A = 9\pi\: u^2}
\end{align*}

\noindent \textbf{Ejercicio 17:} Calcular el área de la región exterior a la circunferencia cuya ecuación polar es $r=3$ e interior a la cardioide de ecuación polar $r=3(1+\cos\,\theta)$.

\vspace{5mm}

\noindent \textbf{Solución}.

\vspace{3mm}

\noindent Para calcular el área de la región emplearemos una doble integral en coordenadas polares, tal que:

\begin{equation}\label{eqn:17-1}\tag{1}
	\iint\limits_R\, \mathrm{d}A = \iint\limits_R\, r\,\mathrm{d}r\,\mathrm{d}\theta
\end{equation}

\noindent Para establecer los límites de integración utilizamos las dos curvas dadas, las cuales dependen de $r$ de tal forma que tenemos: $\boldsymbol{3<r<3(1+\cos\,\theta)}$. Ahora bien, para obtener los límites respecto a $\theta$ igualamos las dos ecuaciones para obtener los puntos de intersección:

\begin{align*}
	\cancel{3} = \cancel{3}(1 + \cos\,\theta) \implies \cancel{1} = \cancel{1} + \cos\,\theta \implies \cos\,\theta = 0 \implies \begin{cases}
		\theta = \frac{\pi}{2} \\
		\theta = \frac{3\pi}{2}
	\end{cases}
\end{align*}

\noindent Notamos que tenemos un ángulo que podría darnos problemas a la hora de realizar los cálculos, por lo que nos enfocaremos en el primer cuadrante ($0<\theta<\frac{\pi}{2}$) y aprovecharemos la símetria para multiplicar el area obtenida por 2. Ahora sustituimos los límites de integración obtenidos en la ecuación \eqref{eqn:17-1} y resolvemos:

\begin{align*}
	2\Int_{0}^{\frac{\pi}{2}} \Int_{3}^{3(1+\cos\,\theta)}\, r\,\mathrm{d}r\,\mathrm{d}\theta = 2\Int_{0}^{\frac{\pi}{2}}\, \left.\frac{r^2}{2}\right|_{3}^{3(1+\cos\,\theta)} \mathrm{d}\theta = 2\Int_{0}^{\frac{\pi}{2}}\, \left(\frac{9(1+\cos\,\theta)^2}{2}  - \frac{9}{2}\right)  \mathrm{d}\theta
\end{align*}

\begin{align*}
	\implies \cancel{2}\left[ \Int_{0}^{\frac{\pi}{2}}\, \frac{9(1+2\cos\,\theta+\cos^2\,\theta)}{\cancel{2}} \mathrm{d}\theta  - \frac{9}{\cancel{2}}\, \Int_{0}^{\frac{\pi}{2}}\, \mathrm{d}\theta \right]
\end{align*}

\begin{align*}
	\implies 9\Int_{0}^{\frac{\pi}{2}}\, (1+2\cos\,\theta+\cos^2\,\theta)\, \mathrm{d}\theta  - 9 (\frac{\pi}{2})
\end{align*}

\begin{align*}
	\implies 9\Int_{0}^{\frac{\pi}{2}}\, (1+2\cos\,\theta+\cos^2\,\theta)\, \mathrm{d}\theta  - \frac{9\pi}{2}
\end{align*}

\begin{align*}
	\implies \underbrace{9\Int_{0}^{\frac{\pi}{2}}\, \left(1+2\cos\,\theta+\frac{1+\cos(2\theta)}{2}\,\right)\, \mathrm{d}\theta - \frac{9\pi}{2}}_{\text{multiplicamos toda la expresión por $\frac{2}{2}$}}
\end{align*}

\begin{align*}
	\implies \frac{9}{2}\Int_{0}^{\frac{\pi}{2}}\, \left(2+4\cos\,\theta+1+\cos(2\theta)\,\right)\, \mathrm{d}\theta - \frac{18\pi}{4} = \frac{9}{2}\Int_{0}^{\frac{\pi}{2}}\, \left(3+4\cos\,\theta+\cos(2\theta)\,\right)\, \mathrm{d}\theta - \frac{18\pi}{4}
\end{align*}

\begin{align*}
	\implies \frac{9}{2}\,\left.\left(3\theta+4\sen(\theta)+\cancelto{0}{\frac{\sen(2\,\theta)}{2}}\right)\right|_{0}^{\frac{\pi}{2}} - \frac{18\pi}{4}
\end{align*}

\begin{align*}
	\implies \frac{9}{2}\left[3\left(\frac{\pi}{2}\right) + 4\underbrace{\left(\sen(\frac{\pi}{2})-\sen(0)\right)}_{1} \right] - \frac{18\pi}{4}
\end{align*}

\begin{align*}
	\implies \frac{27\pi + 32 - 18}{4} = \frac{9\pi}{4} + 18
\end{align*}

\begin{align*}
	\pmb{\color{red} \therefore\: A = 18\,\frac{9\pi}{4}\: \: u^2}
\end{align*}

\noindent \textbf{Ejercicio 18:} Calcular el área de la región limitada por la lemniscata cuya ecuación en coordenadas polares es $r^2=4\cos(2\theta)$.

\vspace{5mm}

\noindent \textbf{Solución.}

\vspace{3mm}

\noindent Para calcular el área de la lemniscata, emplearemos la siguiente integral doble tomando en cuenta que la ecuación brindada está en coordenadas polares:

\begin{equation}\label{eqn:18-1}\tag{1}
	\iint\limits_R\, \mathrm{d}A = \iint\limits_R\, r\,\mathrm{d}r\,\mathrm{d}\theta
\end{equation}

\noindent Una vez definida la doble integral que ocuparemos, necesitamos obtener los límites de integración. Como podemos observar en la ecuación \eqref{eqn:18-1}, los primeros límites corresponden a los impuestos en $r$. En este caso el valor del radio de una lemniscata va de $0$ al de la ecuación de la misma, por lo que debemos despejar a $r$:

\begin{align*}
	r^2 = 4\cos(2\theta) \implies r = 2\sqrt{\cos(2\theta)}
\end{align*}

\noindent Ahora, para el rango de valores de $\theta$ nos basaremos en el calculo de su primer cuadrante y multiplicaremos el area obtenida por 4. En cualquier lemniscata, el rango de $\theta$ en el primer cuadrante es $\,0<\theta<\frac{\pi}{4}$. Teniendo los dos límites de la integral doble, procedemos a resolverla:

\begin{align*}
	\Int_{0}^{\frac{\pi}{4}} \Int_{0}^{2\sqrt{\cos(2\theta)}}\, r\,\mathrm{d}r\,\mathrm{d}\theta = \Int_{0}^{\frac{\pi}{4}} \left.\frac{r^2}{2}\right|_{0}^{2\sqrt{\cos(2\theta)}}\,\mathrm{d}\theta = \Int_{0}^{\frac{\pi}{4}}\,\frac{(2\sqrt{\cos(2\theta)})^2}{2}\,\mathrm{d}\theta = \Int_{0}^{\frac{\pi}{4}}\, \frac{4\cos(2\theta)}{2}\,\mathrm{d}\theta
\end{align*}

\begin{align*}
	\implies 2\,\Int_{0}^{\frac{\pi}{4}}\, \cos(2\theta)\,\mathrm{d}\theta = \cancel{2}\,\left.\frac{\sen(2\theta)}{\cancel{2}}\right|_{0}^{\frac{\pi}{4}} = \left(\sen(\frac{\pi}{2}) - \cancelto{0}{\sen(0)}\right) = 1
\end{align*}

\noindent Tenemos que el valor del área del primer cuadrante de la lemniscata es de 1, por lo que dicho valor es multiplicado por los 4 cuadrantes que conforman esta figura y obtenemos:

\begin{align*}
	\pmb{\color{red} \therefore\: A = 4\: \: u^2}
\end{align*}




\noindent \textbf{Ejercicio 27:} Utilizar el Teorema de Green para calcular el área de la región cerrada que es limitada por la elipse de ecuación $9x^2 + 4y^2 = 36$.

\vspace{5mm}

\noindent \textbf{Solución.}

\vspace{3mm}

\noindent Se sabe que calcular el área por integral doble equivale a integrar la función $f(x, y) = 1$. De esta forma, el integrando puede trabajarse como:

\begin{equation}\label{eqn:27-1}\tag{1}
	f(x, y) = \frac{\partial Q}{\partial x} - \frac{\partial P}{\partial y} \implies 1 = 1 - 0
\end{equation}

\noindent De la ecuación \eqref{eqn:27-1} obtenemos que $Q_x = 1$ y $P_y = 0$. Estas funciones se calculan a partir de la integración:

\begin{align*}
	Q_x = 1 \implies \int Q_x\, \mathrm{d}x = \int \mathrm{d}x \implies Q = x + C_1 \\
	P_y = 0 \implies \int P_y\, \mathrm{d}y = \int 0\, \mathrm{d}y \implies P = C_2 \\
\end{align*}

\noindent Como $C_1$ y $C_2$ son constantes arbitrarias, para el cálculo del área asumiremos que ambas son 0. Por lo tanto, la integral de línea a calcular es:

\begin{align*}
	\iint_R\,\mathrm{d}A = \oint_C\, P\,\mathrm{d}x + Q\,\mathrm{d}y = \oint_C\, x\, \mathrm{d}y
\end{align*}

\noindent Ahora debemos parametrizar la elipse dada, por lo que reescribimos la ecuación de forma estándar y posteriormente parametrizamos:

\begin{align*}
	9x^2 + 4y^2 = 36 \implies \frac{9x^2 + 4y^2}{36} = \frac{36}{36} \implies \frac{x^2}{4} + \frac{y^2}{9} = 1
\end{align*}

\begin{align*}
	\implies \begin{cases}
		x = 2\cos(t) \implies \mathrm{d}x = -2\sen(t)\\
		y = 3\sen(t) \implies \mathrm{d}y = 3\cos(t)\\
	\end{cases} \implies \underbrace{0 \leq t \leq 2\pi}_{\text{Se trata de una elipse cerrada.}}
\end{align*}

\noindent Procedemos a calcular la integral de línea a partir de la parametrización y los límites obtenidos:

\begin{align*}
	\oint_C\, x\,\mathrm{d}y = \Int_{0}^{2\pi}\, (2\cos t)(3\cos t)\,\mathrm{d}t = \Int_{0}^{2\pi}\, (6\cos^2 t)\,\mathrm{d}t
\end{align*}

\begin{align*}
	= 6\Int_{0}^{2\pi}\, \frac{1+\cos(2t)}{2},\mathrm{d}t
	\implies \frac{6}{2}\Int_{0}^{2\pi}\, 1+\cos(2t)\,\mathrm{d}t
\end{align*}

\begin{align*}
	= 3\,\left.\left(t + \frac{\sen(2t)}{2}\right)\right|_{0}^{2\pi} = 3\left[ \left(2\pi - 0\right) + \left( \cancelto{0}{\frac{\sen(4\pi)}{2}} - \cancelto{0}{\frac{\sen(0)}{2}} \right) \right] = 3(2\pi)
\end{align*}

\begin{align*}
	\pmb{\color{red} \therefore\: A = 6\pi\: \: u^2}
\end{align*}

\end{document}

 
