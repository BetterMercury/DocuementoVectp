\documentclass[12pt]{article}
\usepackage[spanish]{babel}
\usepackage{graphicx}
\usepackage{amsmath}
\usepackage{amssymb}
\usepackage{cancel}
\usepackage{hyperref}
\usepackage{xcolor}
\graphicspath{ {Imagenes/} }

\pagestyle{empty}
\topmargin = 0pt
\headheight= 0pt
\headsep=0pt
\oddsidemargin= 25pt
\textwidth=420pt

\DeclareMathOperator{\arcsec}{arcsec}
\DeclareMathOperator{\arccot}{arccot}
\DeclareMathOperator{\arccsc}{arccsc}

\title{Serie 4: Cálculo Vectorial}
\author{López, Ricardo}
\begin{document}
	
\newcommand{\Int}{\int\limits}

\noindent \textbf{Ejercicio 6:} Por medio de la integral doble, calcular el área de la región localizada entre las curvas de ecuación $2y^2=x-2$, $x^2-4y^2=4$, $x=4$.

\vspace{5mm}

\noindent \textbf{Solución.}

\vspace{5mm}

\noindent Despejamos las ecuaciones de las curvas:

\begin{equation}\tag{1}
	2y^2 = x-2 \implies y^2 = \frac{x}{2}-1 \implies y = \sqrt{\frac{x}{2}-1}
\end{equation}

\begin{equation}\tag{2}
	x^2-4y^2=4 \implies 4y^2=x^2-4 \implies y^2=\frac{x^2}{4}-1 \implies \boldsymbol{y=\sqrt{\frac{x^2}{4}-1}} 
\end{equation}

\noindent Obtenemos el valor de $x$ igualando ambas ecuaciones y resolviendo:

\begin{equation}\tag{3}
	\sqrt{\frac{x}{2}-1}=\sqrt{\frac{x^2}{4}-1} \implies \frac{x}{2}-\cancel{1}=\frac{x^2}{4}-\cancel{1} \implies 2x = x^2 \therefore \boldsymbol{x=2}
\end{equation}

\noindent Calculamos la doble integral para obtener el area:

\begin{equation}\tag{4}
	\Int_2^4 \Int_{\sqrt{\frac{x}{2}-1}}^{\sqrt{\frac{x^2}{4}-1}} \, \mathrm{d}y \,\mathrm{d}x \implies \Int_2^4 \left(\sqrt{\frac{x}{2}-1} - \sqrt{\frac{x^2}{4}-1}\right)\, \mathrm{d}x
\end{equation}

\noindent Aplicamos linearidad

\begin{equation}\label{eqn:5}\tag{5}
	\frac{1}{2} \Int_2^4 \sqrt{x^2-4}\, \mathrm{d}x - \frac{1}{\sqrt{2}} \Int_2^4 \sqrt{x-2}\, \mathrm{d}x
\end{equation}

\noindent Resolvemos las integrales por separado:

\begin{equation}\label{eqn:6}\tag{6}
	\frac{1}{2} \Int_2^4 \sqrt{x^2-4}\, \mathrm{d}x
\end{equation}

\begin{equation}\label{eqn:7}\tag{7}
	\frac{1}{\sqrt{2}} \Int_2^4 \sqrt{x-2}\, \mathrm{d}x
\end{equation}

\noindent Realizando sustitución trigonométrica:

\begin{align*}
	x = 2 \sec(u) \to u = \arcsec(\frac{x}{2})\, , \ dx=2\sec(u)\tan(u) \mathrm{d}u
\end{align*}

\noindent Sustituimos en \eqref{eqn:6}:

\begin{align*}
	\frac{1}{2} \int 2\sec(u)\sqrt{4\sec^2(u)-4}\tan(u) \mathrm{d}u
\end{align*}

\noindent Simplificamos usando $4\sec^2(u)-4 = 4\tan^2(u)$:

\begin{align*}
	\frac{1}{2}\, 4\int \sec(u)\tan^2(u)\, \mathrm{d}u \implies 2\int \sec(u)\tan^2(u)\, \mathrm{d}u
\end{align*}

\noindent Reescribimos usando identidades trigonométricas: $\tan^2(u)=\sec^2(u)-1$

\begin{align*}
	= 2\int \sec(u)(\sec^2(u)-1)\, \mathrm{d}u \implies = 2\int (\sec^3(u)-\sec(u))\, \mathrm{d}u
\end{align*}

\begin{align*}
	= 2\left(\int \sec^3(u)\, \mathrm{d}u - \int \sec(u)\, \mathrm{d}u\right)
\end{align*}

\noindent Aplicamos la formula de reducción, con $n=3$:

\begin{center}
	$\int \sec^n(u)\, \mathrm{d}u = \frac{n-2}{n-1} \int \sec^{n-2}(u)\, \mathrm{d}u + \frac{\sec^{n-2}(u)\tan(u)}{n-1}$
\end{center}

\begin{align*}
	= 2\left(\frac{\sec(u)\tan(u)}{2} + \frac{1}{2} \int \sec(u)\, \mathrm{d}u - \int \sec(u)\, \mathrm{d}u\right)
\end{align*}

\begin{align*}
	= 2\left(\frac{\sec(u)\tan(u)}{2} - \frac{1}{2}\int \sec(u)\, \mathrm{d}u\right) \implies 2\left(\frac{\sec(u)\tan(u)}{2} - \frac{1}{2}\ln(\tan(u)+\sec(u))\right)
\end{align*}

\begin{align*}
	= \sec(u)\tan(u) - \ln(\tan(u)+\sec(u))
\end{align*}

\noindent Deshacemos la sustitución $u = \arcsec(\frac{x}{2})$, usando:

\begin{align*}
	\tan\left(\arcsec\left(\frac{x}{2}\right)\right)=\sqrt{\frac{x^2}{4}-1}\quad \mathrm{y} \
	\sec\left(\arcsec\left(\frac{x}{2}\right)\right)=\frac{x}{2}
\end{align*}

\begin{align*}
	\left.\frac{x}{2}\sqrt{\frac{x^2}{4}-1}-\ln\left(\sqrt{\frac{x^2}{4}-1}+\frac{x}{2}\right)\right|_2^4
\end{align*}

\begin{align*}
	=\left(\frac{4}{2}\sqrt{\frac{4^2}{4}-1} - \ln(\sqrt{\frac{4^2}{4}-1}+\frac{4}{2})\right) - \left(\frac{2}{2}\cancelto{0}{\sqrt{\frac{2^2}{4}-1}} - \ln(\cancelto{0}{\sqrt{\frac{2^2}{4}-1}}+\cancelto{1}{\frac{2}{2}})\right)
\end{align*}

\begin{align*}
	=\left(2\sqrt{3} - \ln(\sqrt{3}+2)\right) - \cancelto{0}{\ln(1)}
\end{align*}

\noindent Por lo tanto tenemos que el resultado de la ecuación \eqref{eqn:6} es:

\begin{align*}
	=2\sqrt{3} - \ln(\sqrt{3}+2)
\end{align*}

\noindent Ahora pasamos a resolver resolvemos la ecuación \eqref{eqn:7}:

\begin{align*}
	\frac{1}{\sqrt{2}} \Int_2^4 \sqrt{x-2}\, \mathrm{d}x
\end{align*}

\noindent Utilizamos la sustitución $u=x-2 \to \mathrm{d}u = \mathrm{d}x$:

\begin{align*}
	\int \sqrt{u}\, \mathrm{d}u \implies \int u^{\frac{1}{2}}\, \mathrm{d}u \implies \frac{2}{3}u^{\frac{3}{2}} + C
\end{align*}

\noindent Deshacemos la sustitución previa:

\begin{align*}
	\frac{1}{\sqrt{2}}\left[\left.\frac{2}{3}(x-2)^{\frac{3}{2}}\right|_2^4\right] \implies \frac{2}{3\sqrt{2}} \left[(4-2)^{\frac{3}{2}} - \cancelto{0}{(2-2)^{\frac{3}{2}}}\right] \implies \frac{2}{3\sqrt{2}} \left[(2)^{\frac{3}{2}}\right] = \frac{2\sqrt{8}}{3\sqrt{2}} = \frac{4\cancel{\sqrt{2}}}{3\cancel{\sqrt{2}}} = \pmb{\frac{4}{3}}
\end{align*}

\noindent Resolvemos el problema sustituyendo los valores obtenidos de \eqref{eqn:6} y \eqref{eqn:7} en la ecuación \eqref{eqn:5}:

\begin{align}
	\frac{1}{2} \Int_2^4 \sqrt{x^2-4}\, \mathrm{d}x - \frac{1}{\sqrt{2}} \Int_2^4 \sqrt{x-2}\, \mathrm{d}x = 2\sqrt{3} - \ln(\sqrt{3}+2) - \frac{4}{3}
\end{align}

\begin{align*}
	\therefore \pmb{A = 2\sqrt{3} - \ln(\sqrt{3}+2) - \frac{4}{3} u^2}
\end{align*}

\noindent \textbf{Ejercicio 9:} Calcular $\iint\limits_R \frac{x+y}{1+x-y}\, \mathrm{d}A$, donde $R$ es la región limitada por las gráficas de $y=x$, $y=-x$, $y=x-4$ y $y=-x+4$.

\vspace{3mm}

\noindent Sugerencia: Haga la transformación $\begin{cases} u = x+y\\v=x-y\end{cases}$

\noindent \textbf{Solución}.

\vspace{3mm}

\noindent Considerando la transformación sugerida, obtenemos los factores de escala tales que: $\mathrm{d}A = h_u\,h_v\,\mathrm{d}u\,\mathrm{d}v$

\begin{align*}
	\begin{cases}
		u = x+y\\
		v=x-y
	\end{cases} \implies u+v=x+x+\cancel{y}-\cancel{y} \implies 2x = u+v \implies x = \frac{u+v}{2}
\end{align*}

\begin{align*}
	\begin{cases}
		u = x+y\\
		-v = -x+y
	\end{cases} \implies u-v=\cancel{x}-\cancel{x}+y-y \implies 2y = u-v \implies y = \frac{u-v}{2}
\end{align*}

\begin{equation}\label{eqn:9-1}\tag{1}
	\overline{R}(u,v)=\left(\frac{u+v}{2}\right)\hat{i} + (\frac{u-v}{2})\hat{j}
\end{equation}

\noindent Sabiendo que $h_u = |\frac{\partial \overline{R}}{\partial u}|$ y $h_v = |\frac{\partial \overline{R}}{\partial v}|$, entonces:

\begin{align*}
	\frac{\partial \overline{R}}{\partial u} = \frac{1}{2}\hat{i} + \frac{1}{2}\hat{j} \qquad \qquad
	\frac{\partial \overline{R}}{\partial v} = \frac{1}{2}\hat{i} - \frac{1}{2}\hat{j}
\end{align*}

\begin{align*}
	\left|\frac{\partial \overline{R}}{\partial u}\right| = \left|\frac{\partial \overline{R}}{\partial v}\right| = \sqrt{\frac{1}{2}}
\end{align*}

\begin{align*}
	\mathrm{d}A = \sqrt{\frac{1}{2}}\,\sqrt{\frac{1}{2}}\, \mathrm{d}u\,\mathrm{d}v\, \qquad \therefore \qquad \mathrm{d}A = \frac{1}{2}\,\mathrm{d}u\,\mathrm{d}v = \frac{1}{2}\,\mathrm{d}v\,\mathrm{d}u
\end{align*}

\noindent Calculamos los límites de integración a partir de las curvas dadas y la transformación sugerida:

\begin{align*}
	y = x  \implies \cancelto{{\color{blue}u}}{x-y} = 0 \implies {\color{blue}u} = 0 \\
	y = -x \implies \cancelto{{\color{red}v}}{x+y} = 0 \implies {\color{red}v} = 0 \\
	y = x - 4 \implies \cancelto{{\color{blue}u}}{x-y} = 4 \implies {\color{blue}u} = 4 \\
	y = -x + 4 \implies \cancelto{{\color{red}v}}{x+y} = 4 \implies {\color{red}v} = 4 \\
\end{align*}

\noindent Realizamos la integral doble, sustituyendo la transformación donde es posible:

\begin{equation*}
	 \iint\limits_R \frac{x+y}{1+x-y}\, \mathrm{d}A = \Int_0^4\Int_0^4 \frac{u}{1+v}\, \frac{1}{2}\,\mathrm{d}v\,\mathrm{d}u = \frac{1}{2}\,\Int_0^4\Int_0^4 \frac{u}{1+v}\, \mathrm{d}v\,\mathrm{d}u 
\end{equation*}

\noindent Es importante notar que al tener límites de integración iguales en la doble integral, se pueden intercambiar los diferenciales sin mayor complicación. Resolvemos:

\begin{equation}\label{eqn:9-2}\tag{2}
	\frac{1}{2}\,\Int_0^4\Int_0^4 \frac{u}{1+v}\, \mathrm{d}v\,\mathrm{d}u
\end{equation}

\begin{align*}
	= \frac{1}{2}\,\Int_0^4 u \Int_0^4 \frac{\mathrm{d}v}{1+v}\, \mathrm{d}u
	= \frac{1}{2}\,\Int_0^4 u \left.\ln(1+v)\right|_0^4\, \mathrm{d}u = \frac{1}{2}\,\Int_0^4 u( \ln(5)-\cancelto{0}{\ln(1)})\, \mathrm{d}u \\
	= \frac{1}{2}\,\Int_0^4 u \ln(5)\, \mathrm{d}u = \frac{\ln(5)}{2}\,\Int_0^4 u\, \mathrm{d}u = \frac{\ln(5)}{2}\, \left.\frac{u^2}{2}\right|_0^4 = \frac{\ln(5)}{\cancel{4}}\left(4\cancel{^2}-0^2\right)
\end{align*}

\begin{align*}
	\pmb{\therefore \iint\limits_R \frac{x+y}{1+x-y}\, \mathrm{d}A = 4\ln(5)}
\end{align*}

\end{document}

 
